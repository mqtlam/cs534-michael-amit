\title{CS534 Implementation Assignment 1:\\Linear Regression, Perceptron}
\author{
        Amit Bawaskar, Michael Lam\\
        EECS, Oregon State University\\
        %\email{}
        %\and
}

\documentclass[12pt]{article}
\usepackage[english]{babel}
\usepackage{graphicx}
\usepackage{subfig}
\usepackage{amsmath}
\usepackage{hyperref}
\hypersetup{
    colorlinks,%
    citecolor=green,%
    filecolor=magenta,%
    linkcolor=red,%
    urlcolor=cyan
}

%\underset{x}{\operatorname{argmax}}
%\underset{x}{\operatorname{argmin}}
\DeclareMathOperator*{\argmin}{arg\,min}
\DeclareMathOperator*{\argmax}{arg\,max}

\begin{document}
\maketitle

\begin{abstract}
In this assignment, we implemented two versions of linear regression and perceptron, and compared their performance.
\end{abstract}

% -------------------------------------------------
\section{Introduction}
We implemented linear regression to solve the regression problem. Two versions were implemented and compared for performance: batch gradient descent and stochastic gradient descent. We also implemented two different versions of perceptron to solve the binary classification problem: batch perceptron and voted perceptron. Voted perceptron provides better behavior for the non-linearly separable case of training data.

\section{Linear Regression}
This section compares the performance of our batch gradient descent and stochastic gradient descent algorithms for linear regression.

\subsection{Batch Gradient Descent}
(Report weight vector for each dataset, SSE for each test set, plot convergence)

\subsection{Stochastic Gradient Descent}
(Report weight vector for each dataset, SSE for each test set, plot convergence)

\section{Perceptron}
This section presents results from our batch perceptron and voted perceptron.

\subsection{Batch Perceptron}
(Report plot of classification error on training set, scatter plot of training data)

\subsection{Voted Perceptron}
(Report plot of classification error on training set, visualize decision boundary, report \(w_{avg}\))

\section{Discussion}
Discussion here

\end{document}
This is never printed
